\documentclass{oci}
\usepackage[utf8]{inputenc}
\usepackage{lipsum}

\title{Paltas}

\begin{document}
\begin{problemDescription}
Francisca es fanática de las paltas.
Las come al desayuno, al almuerzo y a la cena.
Con pan, en una ensalada o con leche.
A veces simplemente pela una palta y se las come a mascadas.
Cada vez que va a la feria compra una malla entera para así tener suficientes paltas.

Comprar paltas es un proceso complejo.
Dependiendo de cuándo quieres comerlas, debes elegir entre comprar paltas lo suficientemente maduras
que estén listas para consumir o paltas que aún no han madurado lo suficiente para poder consumirlas
en un par de días más.
Más aún, como cualquier otra fruta, algunas paltas maduran más rápido que otras.
Cada vez que va a la feria, Francisca tiene que pensar en todo esto y necesita un poco de ayuda
para poder comprar paltas de forma más eficiente.

Francisca quiere comprar una malla de $n$ paltas.
Enumeraremos cada una de las paltas en la malla con un entero entre $1$ a $n$.
Además, modelaremos el \emph{estado de maduración} de una palta con un entero mayor o igual que cero,
donde 0 representa el estado al momento de comprar la malla y cualquier entero mayor que 0
indica un grado de madurez mayor.
Es decir, en un inicio todas las paltas en la malla tienen un estado de maduración igual a 0 y
el valor irá aumentando con el tiempo.

Cada palta madura a una velocidad distinta.
Denotaremos con $r_i$ la \emph{tasa de maduración} de la palta $i$-ésima.
Es decir, la palta $i$-ésima madura $r_i$ unidades por cada unidad de tiempo.
En otras palabras, en el tiempo $t$, la palta $i$-ésima tiene un estado de maduración
igual a $t \cdot r_i$.

Una palta solo puede consumirse si su estado de maduración se encuentra en el rango adecuado.
Específicamente, una palta puede consumirse si su estado de maduración es mayor o igual que
$k_{min}$ y menor o igual que $k_{max}$.
Si el estado de maduración es menor que $k_{min}$, la palta aún está muy verde.
Si el estado es mayor que $k_{max}$, la palta ya está podrida.

Dadas las tasas de descomposición de las $n$ paltas, tu trabajo es responder a $q$ consultas
del siguiente tipo:
¿Cuántas paltas pueden consumirse en el tiempo $x$?
\end{problemDescription}

\begin{inputDescription}

La primera línea contiene 4 enteros
$n$, $q$, $k_{min}$ y $k_{max}$
que indican la cantidad de paltas, el número de consultas y los límites del rango de madurez aceptado.
Los valores satisfacen las siguientes restricciones:

\begin{itemize}
  \item $1 \leq n \leq 10^5$
  \item $1 \leq q \leq 10^5$
  \item $0 \leq k_{min} \leq k_{max} \leq 10^9$
  \item $x \leq 10^9$
\end{itemize}

La segunda línea contiene $n$ enteros.
El entero $i$-ésimo corresponde a la tasa de descomposición $r_i$
de la palta $i$-ésima.

Luego vienen $q$ líneas describiendo la consultas.
Cada línea contiene un entero $x$.
La respuesta a la consulta debe ser cuántas paltas pueden consumirse
en el tiempo $x$.

\end{inputDescription}

\begin{outputDescription}
La salida debe contener $q$ líneas con la respuesta a cada una de las consultas en el orden en que
fueron entregadas.
\end{outputDescription}

\begin{scoreDescription}
  \subtask{35}
  Se probarán varios casos en donde $n, q \leq 10^3$.
  \subtask{65}
  Se probarán varios casos sin restricciones adicionales.
\end{scoreDescription}

\begin{sampleDescription}
\sampleIO{sample-1}
\begin{center}
\begin{minipage}{0.95\textwidth}
\textbf{Explicación:} La primera consulta pregunta por cuantas paltas pueden consumirse en el tiempo 3.
En este tiempo, las paltas tendrán estados de maduración 3, 6, 9 y 12.
Como el rango de madurez aceptado es $[6, 10]$, solo dos paltas pueden consumirse (las paltas 2 y 3).
La siguiente consulta pregunta por el tiempo 7.
En este tiempo, las paltas tendrán estados de maduración 7, 14, 21 y 28.
Por tanto, solo una palta puede consumirse (la palta 1).
En la última consulta se pregunta por el tiempo 12.
En este tiempo las paltas tendrán estados de maduración 12, 24, 36 y 48.
Por tanto, ninguna palta puede consumirse.
\end{minipage}
\end{center}

\sampleIO{sample-2}
\begin{center}
\begin{minipage}{0.95\textwidth}
\textbf{Explicación:}
La única consulta pregunta por el tiempo 5.
En este tiempo los estados de maduración serán 5, 5, 5, 5 y 5.
Como el rango permitido es $[5, 5]$ todas las paltas pueden consumirse.
\end{minipage}
\end{center}
\end{sampleDescription}

\end{document}
