\documentclass{oci}
\usepackage[utf8]{inputenc}
\usepackage{lipsum}

\title{Paltas}

\begin{document}
\begin{problemDescription}
Todos hemos ido alguna vez a la feria a comprar paltas, ¿no?
Cuando uno va a comprar paltas, dependiendo de cuanto hambre y paciencia tenga, uno puede escoger entre comprar paltas más maduras para comer el mismo día, o paltas que aún no están maduras para un par de días (o semanas) más adelante.

Al igual que el resto de frutas, hay algunas paltas que se descomponen más rápido que otras. Modelaremos el estado de descomposición de las paltas como un natural $d_i$, donde 0 representa el estado de la palta justo después de comprarla y cualquier entero mayor a 0, indica cierto grado de madurez/descomposición.

Supongamos que compramos una malla de $n$ paltas donde todas se encuentran en el mismo estado inicial $d_i = 0$ para todo $i \in \{1, ..., n\}$.

Si denotamos $r_i$ como la tasa de descomposición de cada palta, podemos decir que en el tiempo $t$, la palta $i$-ésima presenta una descomposición de $t \cdot r_i$. Bajo este modelo, una palta se considera madura (se puede consumir) si su nivel de descomposición se encuentra dentro del rango $[k_{min}, k_{max}]$. Si es menor, la palta no se podrá consumir debido a que aún no madura y si es mayor, la palta se considerará podrida y por motivos sanitarios tampoco será posible consumirla.

Dadas las tasas de descomposición de $n$ paltas, tu trabajo es responder a $q$ preguntas del siguiente tipo:
¿Cuántas paltas se pueden consumir en el tiempo $x$?
\end{problemDescription}

\begin{inputDescription}
La primera línea contiene 4 enteros $n$, $q$, $k_{min}$ y $k_{max}$ ($1 \leq n, q \leq 10^5$,  $0 \leq k_{min} \leq k_{max} \leq 10^9$,  $x \leq 10^9$) que indican la cantidad de paltas, el número de preguntas y los límites del rango de madurez aceptado.\\


La segunda línea contiene $n$ enteros, las tasas de descomposición $r_i$. ($i \in \{1, \dots, n\}$)\\

Luego vienen $q$ líneas, correspondientes a cada consulta $x_i$ ($i \in \{1, \dots, q\}$).
\end{inputDescription}

\begin{outputDescription}
Imprimir $q$ líneas. La $i$-ésima de ellas, deberá contener el número de paltas dentro del rango $[k_{min}, k_{max}]$ en el tiempo $x_i$.
\end{outputDescription}

\begin{scoreDescription}
  \subtask{20}
  Se probarán varios casos en donde $n, q \leq 10^3$.
  \subtask{50}
  Se probarán varios casos sin restricciones adicionales.
\end{scoreDescription}

\begin{sampleDescription}
\sampleIO{sample-1}
\sampleIO{sample-2}
\end{sampleDescription}

\end{document}