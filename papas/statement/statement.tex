\documentclass{oci}
\usepackage[utf8]{inputenc}
\usepackage{lipsum}

\title{Papas fritas}

\begin{document}
\begin{problemDescription}
	Julieta y sus amigos fueron a comer papas fritas a su local favorito,
	la Industria Chilena de Papas Caseras (ICPC).
	Al grupo le gusta hacer todo en conjunto, así que compraron una porción
	de papas para compartir entre todos.
	Los amigos de Julieta son personas muy civilizadas y toman turnos para
	sacar papas (algunos incluso dirían que son personas demasiado civilizadas).
	Para mantener el orden, han diseñado un sistema donde cada
	persona saca solo una papa a la vez.

	En total son $n$ personas en el grupo (incluyendo a Julieta) y todos están
	sentados alrededor de una mesa redonda.
	Una de las personas es elegida para sacar la primera papa.
	Luego saca una papa la persona de la derecha y así sucesivamente en una
	ronda alrededor de la mesa.
	Cuando ya todos sacaron una papa, el ciclo se vuelve a repetir hasta que ya no
	queden más papas.
	Dado el orden que eligieron, Julieta es la $k$-ésima persona en sacar una papa
	por primera vez.

	Como las papas fritas son caseras, cada papa puede tener un tamaño radicalmente
	distinto.
	Julieta es una persona muy observadora y ha determinado el tamaño $t_i$ de cada una
	de ellas.
	Dependiendo del tamaño de las papas que saca, su ración será más o menos grande.
	El tamaño de su ración corresponde a la suma de los tamaños de todas
	las papas de que saca.

	Naturalmente, Julieta quiere que su ración sea lo más grande posible.
	Específicamente, si saca papas maximizando su ración, e independiente
	de la elección de sus amigos, quiere saber el tamaño mínimo que tendrá su ración.
	¿Podrías ayudarla?
\end{problemDescription}

\begin{inputDescription}
	La primera línea de entrada contiene tres enteros $p$, $n$ y $k$
	$(1 \leq k \leq p \leq n \leq 10^6$ y $2 \leq n)$, los cuales
	indican respectivamente la cantidad de papas, la cantidad de personas en el grupo
	y la posición de Julieta

	La segunda línea contiene $p$ enteros, donde el entero $i$-ésimo
	corresponde al tamaño $t_i$ ($1 \leq t_i \leq 10^3$) de la papa número $i$.
\end{inputDescription}

\begin{outputDescription}
	La salida debe contener un único entero, el tamaño de la ración que Julieta tiene
	garantizado sacar si sacas las papas de forma óptima.
\end{outputDescription}

\begin{scoreDescription}
	\subtask{??}
  	Se probarán varios casos en que $k = 1$.
  	\subtask{??}
 	Se probarán varios casos en que $n = k = p$.
  	\subtask{??}
  	Se probarán varios casos en que $n \leq 250$.
	\subtask{??}
	Se probarán varios casos sin restricciones adicionales.
\end{scoreDescription}

\begin{sampleDescription}
\sampleIO{sample-1}
\sampleIO{sample-2}
\end{sampleDescription}

\end{document}
