\documentclass{oci}
\usepackage[utf8]{inputenc}
\usepackage{lipsum}

\title{Rey De Los Piratas}

\begin{document}
\begin{problemDescription}
    Guffy es un joven que sueña con convertirse en el "Rey De Los Piratas", un título que ICPC (International
    Contest of Piracy and Computing) otorga a quien gane su competencia anual de piratería. La competencia
    consiste en que la ICPC publica $N$ archivos online para que los participantes descarguen -- de
    manera legal -- la mayor cantidad posible. Quien descargue la mayor cantidad de archivos es coronado como
    el Rey De Los Piratas.

    La ICPC publica sus archivos en forma de \emph{torrent}. A diferencia de una descarga convencional desde
    un servidor, en donde un servidor central controla el envío de datos a todos quienes descarguen el
    archivo, una descarga por \emph{torrent} tiene $S$ "seeds" quienes toman el rol de proveer el archivo
    a quienes lo descargan, llamados "peers".

    Para ganar la competencia Guffy ha decidido priorizar los torrents con la mayor cantidad de seeds y la menor
    cantidad de peers. De esta forma él puede maximizar la velocidad de descarga al tener más proveedores y menos
    competencia. Ayuda a Guffy a elegir qué torrents descargar.
\end{problemDescription}

\begin{inputDescription}
La primera linea de la entra consiste de un solo entero $N$, $0 < N \leq 10000$ representado el número de torrents
entre los cuales Guffy debe elegir. Las siguientes $N$ lineas consisten de dos enteros separados por un espacio
$S_i$, $P_i$ representado la cantidad de seeds y peers respectivamente del torrent $i$. Con $0 \leq S \leq 10000$
y $0 \leq P \leq 10000$.
\end{inputDescription}

\begin{outputDescription}
    La salida consiste de un solo entero $I$, representado el índice del torrent con mayor cantidad de seeds, en
    caso de empate se debe elegir el torrent con la menor cantidad de peers.
    En caso de un segundo empate cualquiera de los índices ganadores sirve.
\end{outputDescription}

\begin{scoreDescription}
Este problema no contiene subtareas. Se dará puntaje proporcional de a cuerdo a la cantidad de
casos de prueba correctos siendo 100 el puntaje máximo.
\end{scoreDescription}

\begin{sampleDescription}
\sampleIO{sample-1}
\sampleIO{sample-2}
\end{sampleDescription}

\end{document}
