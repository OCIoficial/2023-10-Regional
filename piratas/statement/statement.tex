\documentclass{oci}
\usepackage[utf8]{inputenc}
\usepackage{lipsum}

\title{Rey De Los Piratas}

\begin{document}
\begin{problemDescription}
    Guffy es un joven que sueña con convertirse en el \emph{Rey De Los Piratas}, un título que la ICPC
    (International Contest of Piracy and Computing) otorga al ganador de su competencia anual de
    piratería.
    La competencia consiste en que la ICPC publica $N$ archivos online para que los participantes
    descarguen (de manera completamente legal).
    Quién descargue la mayor cantidad de archivos es coronado como el Rey De Los Piratas.

    La ICPC publica sus archivos en forma de \emph{torrent}.
    % A diferencia de una descarga convencional, en donde un servidor central
    % controla el envío de datos a todos quienes descarguen el archivo,
    En una descarga por torrent cada archivo tiene asociado un número de \emph{seeds} quienes
    son los que toman el rol de proveer el archivo a quienes lo descargan.
    Quienes descargan el archivo son llamados \emph{peers}.

    Para ganar la competencia Guffy ha decidido priorizar los torrents con la mayor cantidad
    de seeds y la menor cantidad de peers.
    De esta forma puede maximizar la velocidad de descarga al tener más proveedores y menos
    competencia tratando de descargar el mismo archivo.
    ?`Podrías ayudar a Guffy a elegir qué torrents descargar?
\end{problemDescription}

\begin{inputDescription}
    La primera línea de la entrada contiene un solo entero $N$ ($0 < N \leq 10^4$)
    representado el número de torrents entre los cuales Guffy debe elegir.
    Cada torrent es identificado con un número entre $1$ y $N$.
    Las siguientes $N$ líneas describen los torrents.
    La línea $i$-ésima contiene dos enteros $S_i$ y $P_i$ ($0 \leq S \leq 10^4$ y $0 \leq P \leq 10^4$)
    correspondientes respectivamente a la cantidad de seeds y peers del torrent $i$.
\end{inputDescription}

\begin{outputDescription}
    La salida debe contener un único entero, correspondiente al número
    del torrent con la mayor cantidad de seeds.
    En caso de empate se debe elegir el torrent con la menor cantidad de peers.
    Si hay un segundo empate puede imprimirse el número de cualquiera de los torrents ganadores.
\end{outputDescription}

\begin{scoreDescription}
Este problema no contiene subtareas.
Se dará puntaje proporcional a la cantidad de casos de prueba correctos siendo 100 el puntaje máximo.
\end{scoreDescription}

\begin{sampleDescription}
\sampleIO{sample-1}
\begin{center}
    \begin{minipage}{0.95\textwidth}
    \textbf{Explicación caso de ejemplo 1:} La respuesta es el torrent 3 que tiene 5 seeds y 10 peers,
    ya que es el torrent con la mayor cantidad de seeds y ningún otro torrent tiene la misma cantidad.
    \end{minipage}
\end{center}
\sampleIO{sample-2}
\begin{center}
    \begin{minipage}{0.95\textwidth}
    \textbf{Explicación caso de ejemplo 2:} En este caso los torrents 1 y 4 tienen la mayor cantidad
    de seeds con un total de 10.
    Como el torrent 1 tiene menos peers que el torrent 4, la respuesta es el torrent 1.
    \end{minipage}
\end{center}

\end{sampleDescription}

\end{document}
