\documentclass{oci}
\usepackage[utf8]{inputenc}
\usepackage{lipsum}

\title{Poleras}

\begin{document}
\begin{problemDescription}
Javier va a viajar a la Olimpiada Internacional de Informática (IOI) acompañando al equipo Chileno de este año. En su emoción, junto a Cathy y Manuel
decidieron mandar a hacer poleras oficiales de equipo.

Como todos sabemos, hay $n$ tallas de poleras desde la más pequeña (talla $1$) a la más grande (talla $n$). Para cada talla $i$, Javier mandó
a hacer $a_i$ poleras.

Un par de días antes de iniciar el viaje a Hungría para la IOI, Javier se dio cuenta de que había cometido un gran error: ¡Había registrado
mal las tallas de poleras de cada miembro de la delegación, y en verdad necesitaba $b_i$ poleras de talla $i$ en vez de $a_i$!

Ya era muy tarde para mandar a hacer las poleras de nuevo, y Javier desesperadamente buscó una solución para este problema. En un momento de
inspiración, recordó que la secadora de ropa podía achicar una polera cuando se usaba a mucha potencia.

Javier investigó en internet y descubrió que su modelo de secadora podía achicar una polera a lo más $k$ tallas. Es decir, podía achicar una
polera de talla $s$ a una de talla $s-1, s-2, \dots$, hasta $s-k$.

Pero antes de achicar las poleras, Javier quiere asegurarse de que es posible convertir las poleras que ya compró (de talla $a_i$) en poleras de
tallas de las que necesita $b_i$. Como Javier está muy apurado con los temas del viaje, no tiene el tiempo para determinar esto con certeza.

Javier sabe que los competidores de la OCI son los mejores programadores de Chile a nivel escolar, y por eso decidió pedirte tu ayuda. ¿Puedes
ayudar a Javier con su problema?
\end{problemDescription}

\begin{inputDescription}
La primera línea de la entrada contiene los números enteros $n$ y $k$ $(1 \leq n \leq 10^6, 0 \leq k \leq 10^6)$ separados por espacios, indicando la
cantidad de tallas de poleras y la cantidad máxima de tallas que la secadora puede achicar.

La segunda línea contiene los números enteros $a_1, a_2, \dots, a_n$ $(0 \leq a_i \leq 10^9$) separados por espacios, indicando la cantidad de
poleras de cada talla que tiene Javier.

La tercera línea contiene los números enteros $b_1, b_2, \dots, b_n$ $(0 \leq b_i \leq 10^9)$ separados por espacios, indicando la cantidad de
poleras de cada talla que necesita Javier.
\end{inputDescription}

\begin{outputDescription}
Si es posible que Javier obtenga las tallas que necesita, tu programa debe imprimir \texttt{SI}. En el caso contrario, debe imprimir \texttt{NO}.
\end{outputDescription}

\begin{scoreDescription}
  \subtask{10}
  Se probarán varios casos con $k=0$, es decir, la secadora no puede achicar poleras.
  \subtask{20}
  Se probarán varios casos con $k=1$, es decir, la secadora puede achicar a lo más una talla.
  \subtask{30}
  Se probarán varios casos con $k \leq 100$.
  \subtask{40}
  Se probarán varios casos sin restricciones adicionales.
\end{scoreDescription}

\begin{sampleDescription}
% TODO: Add samples
\sampleIO{sample-1}
\sampleIO{sample-2}
\end{sampleDescription}

\end{document}
