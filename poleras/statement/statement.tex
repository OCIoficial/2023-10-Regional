\documentclass{oci}
\usepackage[utf8]{inputenc}
\usepackage{lipsum}

\title{Poleras}

\begin{document}
\begin{problemDescription}
Javier va a viajar a la Olimpiada Internacional de Informática (IOI)
acompañando al equipo Chileno.
En su emoción, decidió mandar a hacer poleras oficiales del equipo.

Como todos sabemos, las poleras vienen en $n$ tallas, siendo la talla 1 la más pequeña
y la talla $n$ la más grande.
Para cada talla $i$, Javier mandó a hacer $a_i$ poleras de esa talla.

Un par de días antes de iniciar el viaje, Javier se dio
cuenta de un gran error que había cometido.
¡Había registrado mal las tallas de poleras de los miembros de la delegación!
En vez de $a_i$ poleras de talla $i$ necesitaba $b_i$ poleras de esa talla.

Como ya era muy tarde para mandar a hacer las poleras de nuevo, Javier
empezó desesperadamente a buscar una solución alternativa.
En un momento de inspiración, recordó que se puede encoger una polera metiéndola a
la secadora de ropa a alta potencia.

Después de investigar en internet, Javier descubrió que, controlando el tiempo de secado, su
secadora puede encoger una polera hasta $k$ tallas.
Es decir, una polera de talla $s$ puede encogerse a las tallas $s-1, s-2, \dots$, $s-k$.
Naturalmente, una polera de talla menor o igual que $k$ solo puede encogerse hasta la talla 1.
Cada polera puede meterse solamente una vez a la secadora para achicarla.

Antes de iniciar el proceso, Javier quiere asegurarse de que es
posible convertir las poleras que ya compró en poleras de
las tallas que necesita.
Específicamente, quiere saber si para cada talla $i$ es posible terminar
con al menos $b_i$ poleras de esa talla si encoge algunas de las poleras que ya compró.
Por ejemplo, considera un escenario representado por la siguiente tabla, donde la cantidad
de tallas es tres ($n=3$):
\begin{center}
\begin{tabular}{cccc}
   & \footnotesize 1 & \footnotesize2  &  \footnotesize 3 \\
\hline
 $a$ & 10 & 21 & 32 \\
 $b$ & 12 & 20 & 30 \\
\hline
\end{tabular}
\end{center}
La fila $a$ representa las poleras que compró de cada talla y la fila $b$,
la cantidad de poleras que necesita. Si la secadora puede encoger poleras
hasta dos tallas ($k=2$), entonces Javier tiene varias opciones para obtener las poleras
que necesita. Algunas de las opciones son:
\begin{itemize}
    \item Encoger la polera extra de talla 2 en una talla y una de las poleras extra de talla
        3 en dos tallas.
    \item Encoger las dos poleras extra de talla 3 en dos tallas.
    \item Encoger dos poleras de talla 2 en una talla y luego una polera de talla 3 en una talla.
\end{itemize}

Como Javier está muy apurado con los preparativos del viaje, no tiene el tiempo para determinar
con certeza si puede encoger las poleras adecuadamente.
?`Podrías ayudarlo?
\end{problemDescription}

\begin{inputDescription}
La primera línea de la entrada contiene los enteros $n$ y $k$ $(1 \leq n \leq 10^6, 0 \leq k \leq 10^6)$,
que indican la cantidad total de tallas y la cantidad máxima de tallas que la secadora puede achicar, respectivamente.

La segunda línea contiene los números enteros $a_1, a_2, \dots, a_n$ $(0 \leq a_i \leq 10^9$),
que corresponden a la cantidad de poleras de cada talla que compró Javier.

La tercera línea contiene los números enteros $b_1, b_2, \dots, b_n$ $(0 \leq b_i \leq 10^9)$, que indican
la cantidad de poleras de cada talla que necesita Javier.
\end{inputDescription}

\begin{outputDescription}
Si para cada $i$ es posible terminar con al menos $b_i$
poleras de esa talla, tu programa debe imprimir \texttt{SI}.
En caso contrario, debe imprimir \texttt{NO}.
\end{outputDescription}

\begin{scoreDescription}
  \subtask{10}
  Se probará varios casos con $k=0$, es decir, la secadora no puede encoger poleras.
  \subtask{20}
  Se probará varios casos con $k=1$, es decir, la secadora puede encoger poleras a lo más una talla.
  \subtask{30}
  Se probará varios casos con $k \leq 100$.
  \subtask{40}
  Se probará varios casos sin restricciones adicionales.
\end{scoreDescription}

\begin{sampleDescription}
% TODO: Add samples
\sampleIO{sample-1}
\begin{center}
    \begin{minipage}{0.94\textwidth}
    \textbf{Explicación caso de ejemplo 1:} Javier compró 10 poleras de talla 1, 21 de talla 2 y 32 de talla 3,
    pero necesita 12 de talla 1, 20 de talla 2 y 30 de talla 3.
    Javier tiene suficientes poleras de talla 2 y 3, pero le faltan dos de talla 1.
    Como no es posible achicar poleras ($k=0$), es imposible obtener las poleras que le falta y por
    lo tanto la respuesta es \texttt{NO}.
    \end{minipage}
\end{center}

\sampleIO{sample-2}
\begin{center}
    \begin{minipage}{0.94\textwidth}
    \textbf{Explicación caso de ejemplo 2:} Este es el ejemplo explicado en el enunciado.
    \end{minipage}
\end{center}

\sampleIO{sample-3}
\begin{center}
    \begin{minipage}{0.94\textwidth}
    \textbf{Explicación caso de ejemplo 3:} En este caso la respuesta es \texttt{NO}, porque si Javier
    achica algunas poleras para obtener las poleras de talla 1 que le faltan, se quedará sin poleras
    suficientes de talla 2 o 3.
    \end{minipage}
\end{center}
\end{sampleDescription}

\end{document}
